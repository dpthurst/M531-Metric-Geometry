\documentclass[12pt]{amsart}
\usepackage{amssymb}
\usepackage{cite}
\usepackage{array}
\usepackage{booktabs}
\usepackage{mdwtab}
\usepackage{mathtools}
\usepackage{lmodern}
\usepackage{microtype}
\usepackage[T1]{fontenc}
\usepackage[utf8]{inputenc}
\usepackage{hyphenat}
\usepackage{enumitem}
\usepackage{mathabx}
\usepackage{color}
\usepackage[pdftex]{graphicx}
\usepackage[pdftex,margin=1in,lmargin=0.75in,rmargin=1.5in,marginparwidth=1.4in]{geometry}
\usepackage[bookmarks=true, bookmarksopen=true,%
    bookmarksdepth=3,bookmarksopenlevel=2,%
    colorlinks=true,%
    linkcolor=blue,%
    citecolor=blue,%
    filecolor=blue,%
    menucolor=blue,%
    urlcolor=blue]{hyperref}
\hypersetup{pdftitle={Metric Geometry: Class Notes}}
\hypersetup{pdfauthor={Dylan P. Thurston et al.}}
\usepackage{url}
\usepackage{tikz}
\usetikzlibrary{arrows}
\tikzstyle{every picture}=[> = to]
% Style for labels on arrows in commutative diagrams
\tikzset{cdlabel/.style={execute at begin node=$\scriptstyle,execute at end node=$}}
\tikzset{implication/.style={double equal sign distance, -implies}}
\tikzset{biimplication/.style={double equal sign distance, implies-implies}}

% A binary operator with a subscript on both sides (and correct spacing)
% Name stands for subscript-operator-subscript
% Perhaps this should be using \manyindices?
\newcommand{\sos}[3]{\mathbin{{}_{#1}\mathord#2_{#3}}}

% manyindices
% Adapted from code by "bza" in comp.text.tex, Feb. 7, 2006
%% USAGE:
%%
%% \manyindices#1#2#3#4#5
%%
%% #1=lower left index
%% #2=upper left index
%% #3=lower right index
%% #4=upper right index
%% #5=main symbol
\makeatletter
\newcommand\mi@kern[1]{%
  \settowidth\@tempdima{$\mi@obj^{#1}$}
  \kern-\@tempdima
  #1
  \settowidth\@tempdima{$\mi@obj$}
  \kern\@tempdima
}

\newtoks\mi@toksp
\newtoks\mi@toksb
\DeclareRobustCommand{\manyindices}[5]{
  \def\mi@obj{#5}
  \mi@toksp\expandafter{\mi@kern{#2}}
  \mi@toksb\expandafter{\mi@kern{#1}}
  \@mathmeasure4\textstyle{#5_{#1}^{#2}}
  \@mathmeasure6\textstyle{#5_{#3}^{#4}}
  \dimen0-\wd6 \advance\dimen0\wd4
  \@mathmeasure8\textstyle{\hphantom{{}_{#1}^{#2}}#5^{\the\mi@toksp#4}_{\the\mi@toksb#3}}
  \hbox to \dimen0{}{\kern-\dimen0\box8}
}
\makeatother 

% Left sub/super scripts
% \lsup is a temporary definition until something better is worked out
% Use \lsupv if the next argument is vertical
\newcommand{\lsub}[2]{{}_{#1}#2}
\newcommand{\lsup}[2]{{}^{#1}\mskip-.6\thinmuskip#2}
\newcommand{\lsupv}[2]{{}^{#1}#2}
\newcommand{\lsubsup}[3]{\manyindices{#1}{\mskip.6\thinmuskip#2\mskip-.6\thinmuskip}{}{}{\mathord{#3}}}
\newcommand{\lsubsupv}[3]{\manyindices{#1}{\mskip.2\thinmuskip#2\mskip-.2\thinmuskip}{}{}{\mathord{#3}}}

\newcounter{saveenum}

% Read the file, if it exists
\newread\testin
\def\maybeinput#1{
\openin\testin=#1
\ifeof\testin\typeout{Warning: input #1 not found}\else\input#1\fi
\closein\testin
}

\def\mathcenter#1{%
  \vcenter{\hbox{$#1$}}%
}

\def\graph#1{
        \includegraphics{#1}
}

\def\mathgraph#1{
        \mathcenter{\graph{#1}}
}

\def\mfig#1{
        \mathcenter{\includegraphics{#1}}
}

\def\mfigb#1{
        \mathcenter{\includegraphics[trim=-1 -1 -1 -1]{#1}}
}

\newcommand{\DPTtodo}[1]{\todo[color=green!40]{#1}}

\newcommand{\arXiv}[1]{\href{http://arxiv.org/abs/#1}{arXiv:#1}}


% \colon with an optional line break. From https://groups.google.com/forum/#!topic/comp.text.tex/Ts7R4WDTK-M
\renewcommand{\colon}{\nobreak\mskip2mu\mathpunct{}\nonscript
  \mkern-\thinmuskip{:}\allowbreak\mskip6muplus1mu\relax}

%%% Local Variables: 
%%% mode: latex
%%% TeX-master: "main"
%%% End: 

% General use
\newcommand{\RR}{\mathbb R}
\newcommand{\CC}{\mathbb C}
\newcommand{\ZZ}{\mathbb Z}
\newcommand{\QQ}{\mathbb Q}
\newcommand{\PP}{\mathbb P}
\newcommand{\EE}{\mathbb E}
\newcommand{\HH}{\mathbb H}
\newcommand{\NN}{\mathbb N}

\newcommand{\comma}{\mathbin ,}
\newcommand{\conn}{\mathbin \#}
\newcommand{\sltwo}{{{\mathfrak{sl}}_2}}
\renewcommand{\sl}{\mathfrak{sl}}
\newcommand{\gl}{\mathfrak{gl}}
\newcommand{\fg}{{\mathfrak g}}
\newcommand{\eps}{\varepsilon}
\newcommand{\abs}[1]{{\lvert #1 \rvert}}
\newcommand{\norm}[1]{{\lVert #1 \rVert}}
\newcommand{\OneHalf}{{\textstyle\frac{1}{2}}}

% Synonyms for commands I never remember
\newcommand{\isom}{\cong}
\newcommand{\superset}{\supset}
\newcommand{\bigcircle}{\bigcirc}
\newcommand{\contains}{\ni}
\newcommand{\tensor}{\otimes}
\newcommand{\bdy}{\partial}

% Stupid overloading.
\newcommand{\lbracket}{[}
\newcommand{\rbracket}{]}

% More delimiters
\DeclarePairedDelimiter\ceil{\lceil}{\rceil}
\DeclarePairedDelimiter\floor{\lfloor}{\rfloor}

% Various operators.
\DeclareMathOperator{\ad}{ad}
\DeclareMathOperator{\Ad}{Ad}
\DeclareMathOperator{\End}{End}
\DeclareMathOperator{\sign}{sign}
\DeclareMathOperator{\Sym}{Sym}
\DeclareMathOperator{\tr}{tr}
\DeclareMathOperator{\Hom}{Hom}
\DeclareMathOperator{\vol}{vol}
\DeclareMathOperator{\rank}{rank}
\DeclareMathOperator{\im}{im}
\DeclareMathOperator{\Gr}{Gr}

% Linear groups
\DeclareMathOperator{\ISO}{\mathit{ISO}}
\DeclareMathOperator{\SO}{\mathit{SO}}
\DeclareMathOperator{\GL}{\mathit{GL}}
\DeclareMathOperator{\SL}{\mathit{SL}}
\DeclareMathOperator{\PSL}{\mathit{PSL}}

% Special knots
\newcommand{\unknot}{\bigcircle}

% Theorems
\theoremstyle{plain}
\newtheorem{theorem}{Theorem}
\newtheorem{proposition}{Proposition}
\newtheorem{lemma}[proposition]{Lemma}
\newtheorem{corollary}[proposition]{Corollary}
\newtheorem{claim}[proposition]{Claim}
\newtheorem{conjecture}[proposition]{Conjecture}
\newtheorem{observation}[proposition]{Observation}

\theoremstyle{definition}
\newtheorem{definition}[proposition]{Definition}
\newtheorem{exercise}[proposition]{Exercise}
\newtheorem{question}[proposition]{Question}
\newtheorem{problem}[proposition]{Problem}

\theoremstyle{remark}
\newtheorem{example}[proposition]{Example}
%\newtheorem{hint}[proposition]{Hint}
\newtheorem*{remark}{Remark}
%\newtheorem{apology}[proposition]{Apology}
%\newtheorem{warning}[proposition]{Warning}

% Hyphenation.
\hyphenation{Thurs-ton}

% Noting scribes
\newcommand{\scribe}[2]{\marginpar{\small #1\\\texttt{#2}}}

%%% Local Variables: 
%%% mode: latex
%%% TeX-master: "main"
%%% TeX-master: t
%%% End: 


\graphicspath{{draws/}}

\begin{document}
\title[Metric Geometry: Class Notes]{Metric Geometry: Class Notes\\
  \small Fall 2018}

\author[Thurston]{Dylan~P.~Thurston}
\author{members of M531}
\date{\today}

\maketitle

\tableofcontents

\section{Metric spaces}
\label{sec:metric-spaces}

\scribe{Dylan Thurston}{2018-08-20}
\begin{definition}
  A \emph{metric space} is a set~$X$ together with a \emph{distance
    function} $d \colon X \times X \to \RR_{\ge 0} \cup \{\infty\}$
  satisfying the following axioms for all $x,y,z \in X$.
  \begin{itemize}
  \item \textbf{(Reflexive)} $d(x,x) = 0$.
  \item \textbf{(Triangle inequality)} $d(x,z) \le d(x,y) + d(y,z)$.
    (Going straight from $x$ to~$z$ is at least as good as going from
    $x$ to~$y$ and then from $y$ to~$z$.)
  \item \textbf{(Symmetry)} $d(x,y) = d(y,x)$.
  \item \textbf{(Distinguishable)} If $d(x,y) = 0$, then $x = y$.
  \end{itemize}
\end{definition}

\begin{example}
  $\RR^2$, with its standard Euclidean metric $d_{\text{Eucl}}$, is a
  metric space.
\end{example}

\begin{example}
  $\ZZ^2$, with the metric $d_{\text{Eucl}}$ induced as a subset of
  $\RR^2$, is a metric space.
\end{example}

\begin{example}
  $\RR^2$ with the taxicab metric defined by
  \[
    d_{\text{taxi}}\bigl((x_1,x_2),(y_1,y_2)\bigr) = \abs{x_1 - y_1} +
    \abs{x_2 - y_2},
  \]
  is a metric space. In this space, the balls are diamond-shape rather
  than circles.
\end{example}

\begin{example}
  The standard unit sphere $S^2 \subset \RR^3$ with the extrinsic
  metric $d_{\text{extr}}$ induced
  as a subset of $\RR^3$ is a metric space.
\end{example}

If you drop the condition that the metric be distinguishable, you get
a \emph{pseudometric} space. For instance, $\RR^2$ with the metric
that is identically $0$ is a pseudometric space, as is $\RR^2$ with
the metric that only measures distance along the first coordinate.

If you drop the condition that the metric is symmetric, you get an
\emph{asymmetric metric}. These are less standard, despite occurring in
natural situations.

\begin{example}
  An (undirected) graph gives a metric space, where the points are the
  vertices and the distance between two vertices is the minimum number
  of edges you need to cross to get from one vertex to another. A
  \emph{directed} graph gives an asymmetric metric space, where you
  are only allowed to cross edges in the direction of the arrow.
\end{example}

Another real-world example of an asymmetric metric is travel
difficulty; it is much harder to walk from the bottom of a mountain to
the top than the other way around.

A metric naturally gives you a \emph{topology}, for instance defining
continuous maps.
\begin{definition}
  If $X$ and $Y$ are metric spaces, a function $f \colon X \to Y$ is
  \emph{continuous} if
  \[
  \forall x \in X, \,\forall \epsilon > 0,\,
  \exists \delta > 0,\,\forall y \in X\colon
  d_X(x,y) < \delta \Rightarrow d_Y(f(x),f(y)) < \epsilon
  \]
  or, loosely, if points sufficiently close to each other in~$X$ are
  close in~$Y$. Subtle variations yield different concepts; for
  instance, if you move the clause ``$\forall y \in X$'' to the
  beginning, then you get the notion of \emph{uniform continuity},
  which depends on the metric and not just on the topology.
\end{definition}

\section{Quasi-isometries (preview)}
\label{sec:quasi-isom-1}
A metric gives a lot more structure than just a topology. On the other
end of the spectrum of what a metric gives you, we have the notion of
\emph{quasi-isometric embedding}, which ignores what happens on the
small scale (which is the only thing relevant to the topology), and in
stead focuses on the behavior on grand scales.

\begin{definition}
  If $X$ and $Y$ are metric spaces, a function $f \colon X \to Y$ is a
  \emph{quasi-isometric embedding} if there are constants $K >1$ and
  $C > 0$ so that $\forall x,y \in X$,
  \[
    \frac{d_X(x,y) - C}{K}
      \le d_Y(f(x), f(y)) \le
        K d_X(x,y) + C.
  \]
  That is, distances in $X$ are distorted by $f$ by at most a
  multiplicative factor~$K$ and an additive constant~$C$.
\end{definition}

\begin{example}
  The identity map
  $(\RR^2, d_{\text{Eucl}}) \to (\RR^2, d_{\text{taxi}})$ is a
  quasi-isometry with $K = \sqrt{2}$ and $C=0$, as is the identity map
  going the other way.
\end{example}

\begin{example}
  The inclusion map
  $(\ZZ^2,d_{\text{Eucl}}) \hookrightarrow (\RR^2,d_{\text{Eucl}})$ is a
  quasi-isometry with $K=1$ and $C=0$, i.e., an \emph{isometry}.
\end{example}

\begin{example}
  The map $f \colon (\RR^2, d_{\text{Eucl}}) \to (\ZZ^2,
  d_{\text{Eucl}})$ defined by
  \[
    f\bigl((x_1,x_2)\bigr) = (\ceil{x_1}, \ceil{x_2})
  \]
  is a quasi-isometry with $K=1$ and $C=2$. (Can this value of $C$ be
  improved?) Note that $f$ is not continuous.
\end{example}

\begin{example}
  There is a quasi-isometry from $(S^2, d_{\text{extr}})$ to
  $(\RR^2, d_{\text{Eucl}})$; for instance, project on to the plane, or
  more simply map all of $S^2$ to a single point. However, there is no
  quasi-isometry in the other direction.
\end{example}

\begin{exercise}
  Verify the above assertions.
\end{exercise}

\begin{definition}
  A quasi-isometric embedding $f \colon X \to Y$ is a
  \emph{quasi-isometry} if every point in~$Y$ is near the image
  of~$X$, or, precisely, if there is a constant $C_2 > 0$ so that
  \[
    \forall y \in Y,\, \exists x \in X\colon d_Y(f(x),y) < C_2.
  \]
\end{definition}

\begin{exercise}
  Show that if $f \colon X \to Y$ is a quasi-isometry, then there is
  also a quasi-isometry $g \colon Y \to X$. What can you
  say about the composition $g \circ f$?
\end{exercise}

\begin{exercise}
  Show that a metric space $(X,d)$ is quasi-isometric to a point iff
  $d$ is bounded.
\end{exercise}

\begin{exercise}
  If $X$ is any metric space, show that any map $f \colon X \to X$
  that moves each point a (globally) bounded amount is a
  quasi-isometry. (For instance, $f$ need not be continuous.)
\end{exercise}

The notion of quasi-isometry is quite useful in group theory, since it
can give us some notion of large-scale geometry even on discrete sets
like a fundamental group. For instance, the following theorem is true.

\begin{theorem}
  If $X$ is a compact manifold,
  then its fundamental group $\pi_1(X)$ is quasi-isometric to its
  universal cover $\wt X$.
\end{theorem}

Here, we use the metric on $\pi_1(X)$ coming from its \emph{Cayley
  graph}, which we will come back to later.

\begin{example}
  For $X = S^1$, we have seen that $\pi_1(S^1) = \ZZ$ is
  quasi-isometric to $\widetilde{S^1} = \RR$. For $X = S^2$, we have
  seen that $\pi_1(S^2) = \{e\}$ is quasi-isometric to
  $\widetilde{S^2} = S^2$.
\end{example}

\begin{question}
  Is there an unbounded metric $d$ on $\RR^2$, inducing the standard
  topology, so that $(\RR^2,d)$ is not quasi-isometric to
  $(\RR^2,d_{\text{Eucl}})$?
\end{question}

\begin{question}
  What properties of metric spaces are preserved by quasi-isometries?
\end{question}

We will consider quasi-isometry and other ``quasi'' notion more later
in the course. In the meantime, you can think about the questions
above as we go along.

\section{Lengths of curves}
\label{sec:lengths}

\begin{definition}
  A \emph{curve}~$\gamma$ in a metric space $(X,d)$ is a continuous map
  \[
    \gamma\colon [0,1] \to X,
  \]
  using the standard metric on the interval.
\end{definition}

What should the length of a curve be? For sufficiently nice curves in
$(\RR^2, d_{\text{Eucl}})$, we can use the calculus definition and set
the length to be
\begin{equation}
  \label{eq:length-eucl}
  \int_0^1 \abs{\gamma'(t)}\,dt.
\end{equation}
However, this doesn't work if the target space has no notion of
differentiation. It doesn't even always work well in $\RR^2$.

\begin{example}
  Take $\gamma$ to be a nice curve in $\RR^2$ (say a semi-circle), and
  let $\phi \colon [0,1] \to [0,1]$ be a weakly monotonic
  reparametrization function (with $\phi(0) = 0$ and $\phi(1) = 1$),
  so that its derivative is almost everywhere~$1/2$ (like the
  \href{https://en.wikipedia.org/wiki/Cantor_function}{Devil's
    Staircase}, with an additional linear function added). Then we
  expect $\gamma \circ \phi$ to describe the same geometric curve as
  $\gamma$ and have the same length. But the
  integral~(\ref{eq:length-eucl}) (to the extent it exists) gives a
  different answer.%
  \footnote{In class, I suggested taking $\phi$ to be a monotonic,
    nowhere-differentiable function, but no such functions exist.}
\end{example}

To get something more general, we look at subdivisions.

\begin{definition}\label{def:length}
  If $\gamma$ is a curve in a metric space~$X$, its \emph{length} is
  defined to be
  \begin{equation}\label{eq:length}
    \ell(\gamma) = \sup \sum_{i=0}^k d(\gamma(t_i), \gamma(t_{i+1})),
  \end{equation}
  where the supremum runs over all subdivisions of $[0,1]$ into
  a finite number of subintervals at
  \[
    0 = t_0 < t_1 < \cdots < t_k < t_{k+1} = 1.
  \]
\end{definition}

Note that if we make a subdivision finer by adding an extra point, the
supremand in~(\ref{eq:length}) can only increase (by the triangle
inequality). This helps with convergence.

\begin{exercise}
  If you know about the theory of
  \href{https://en.wikipedia.org/wiki/Net_(mathematics)}{nets}, verify
  that subdivisions of the interval form a directed set under refining
  subdivisions, and that supremum can be replaced by limit of a net
  in Equation~(\ref{eq:length}).
\end{exercise}

\end{document}

%%% Local Variables: 
%%% mode: latex
%%% TeX-master: t
%%% End: 
