% A binary operator with a subscript on both sides (and correct spacing)
% Name stands for subscript-operator-subscript
% Perhaps this should be using \manyindices?
\newcommand{\sos}[3]{\mathbin{{}_{#1}\mathord#2_{#3}}}

% manyindices
% Adapted from code by "bza" in comp.text.tex, Feb. 7, 2006
%% USAGE:
%%
%% \manyindices#1#2#3#4#5
%%
%% #1=lower left index
%% #2=upper left index
%% #3=lower right index
%% #4=upper right index
%% #5=main symbol
\makeatletter
\newcommand\mi@kern[1]{%
  \settowidth\@tempdima{$\mi@obj^{#1}$}
  \kern-\@tempdima
  #1
  \settowidth\@tempdima{$\mi@obj$}
  \kern\@tempdima
}

\newtoks\mi@toksp
\newtoks\mi@toksb
\DeclareRobustCommand{\manyindices}[5]{
  \def\mi@obj{#5}
  \mi@toksp\expandafter{\mi@kern{#2}}
  \mi@toksb\expandafter{\mi@kern{#1}}
  \@mathmeasure4\textstyle{#5_{#1}^{#2}}
  \@mathmeasure6\textstyle{#5_{#3}^{#4}}
  \dimen0-\wd6 \advance\dimen0\wd4
  \@mathmeasure8\textstyle{\hphantom{{}_{#1}^{#2}}#5^{\the\mi@toksp#4}_{\the\mi@toksb#3}}
  \hbox to \dimen0{}{\kern-\dimen0\box8}
}
\makeatother 

% Left sub/super scripts
% \lsup is a temporary definition until something better is worked out
% Use \lsupv if the next argument is vertical
\newcommand{\lsub}[2]{{}_{#1}#2}
\newcommand{\lsup}[2]{{}^{#1}\mskip-.6\thinmuskip#2}
\newcommand{\lsupv}[2]{{}^{#1}#2}
\newcommand{\lsubsup}[3]{\manyindices{#1}{\mskip.6\thinmuskip#2\mskip-.6\thinmuskip}{}{}{\mathord{#3}}}
\newcommand{\lsubsupv}[3]{\manyindices{#1}{\mskip.2\thinmuskip#2\mskip-.2\thinmuskip}{}{}{\mathord{#3}}}

\newcounter{saveenum}

% Read the file, if it exists
\newread\testin
\def\maybeinput#1{
\openin\testin=#1
\ifeof\testin\typeout{Warning: input #1 not found}\else\input#1\fi
\closein\testin
}

\def\mathcenter#1{%
  \vcenter{\hbox{$#1$}}%
}

\def\graph#1{
        \includegraphics{#1}
}

\def\mathgraph#1{
        \mathcenter{\graph{#1}}
}

\def\mfig#1{
        \mathcenter{\includegraphics{#1}}
}

\def\mfigb#1{
        \mathcenter{\includegraphics[trim=-1 -1 -1 -1]{#1}}
}

\newcommand{\DPTtodo}[1]{\todo[color=green!40]{#1}}

\newcommand{\arXiv}[1]{\href{http://arxiv.org/abs/#1}{arXiv:#1}}


% \colon with an optional line break. From https://groups.google.com/forum/#!topic/comp.text.tex/Ts7R4WDTK-M
\renewcommand{\colon}{\nobreak\mskip2mu\mathpunct{}\nonscript
  \mkern-\thinmuskip{:}\allowbreak\mskip6muplus1mu\relax}

%%% Local Variables: 
%%% mode: latex
%%% TeX-master: "main"
%%% End: 
