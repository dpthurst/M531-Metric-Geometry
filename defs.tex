% General use
\newcommand{\RR}{\mathbb R}
\newcommand{\CC}{\mathbb C}
\newcommand{\ZZ}{\mathbb Z}
\newcommand{\QQ}{\mathbb Q}
\newcommand{\PP}{\mathbb P}
\newcommand{\EE}{\mathbb E}
\newcommand{\HH}{\mathbb H}
\newcommand{\NN}{\mathbb N}
\renewcommand{\epsilon}{\varepsilon}

\newcommand{\comma}{\mathbin ,}
\newcommand{\conn}{\mathbin \#}
\newcommand{\sltwo}{{{\mathfrak{sl}}_2}}
\renewcommand{\sl}{\mathfrak{sl}}
\newcommand{\gl}{\mathfrak{gl}}
\newcommand{\fg}{{\mathfrak g}}
\newcommand{\OneHalf}{{\textstyle\frac{1}{2}}}
\newcommand{\wt}[1]{\widetilde{#1}}

% Synonyms for commands I never remember
\newcommand{\isom}{\cong}
\newcommand{\superset}{\supset}
\newcommand{\bigcircle}{\bigcirc}
\newcommand{\contains}{\ni}
\newcommand{\tensor}{\otimes}
\newcommand{\bdy}{\partial}

% Stupid overloading.
\newcommand{\lbracket}{[}
\newcommand{\rbracket}{]}

% More delimiters
\DeclarePairedDelimiter\ceil{\lceil}{\rceil}
\DeclarePairedDelimiter\floor{\lfloor}{\rfloor}
\DeclarePairedDelimiter\abs{\lvert}{\rvert}
\DeclarePairedDelimiter\norm{\lVert}{\rVert}

% Various operators.
\DeclareMathOperator{\ad}{ad}
\DeclareMathOperator{\Ad}{Ad}
\DeclareMathOperator{\End}{End}
\DeclareMathOperator{\sign}{sign}
\DeclareMathOperator{\Sym}{Sym}
\DeclareMathOperator{\tr}{tr}
\DeclareMathOperator{\Hom}{Hom}
\DeclareMathOperator{\vol}{vol}
\DeclareMathOperator{\rank}{rank}
\DeclareMathOperator{\im}{im}
\DeclareMathOperator{\Gr}{Gr}

% Linear groups
\DeclareMathOperator{\ISO}{\mathit{ISO}}
\DeclareMathOperator{\SO}{\mathit{SO}}
\DeclareMathOperator{\GL}{\mathit{GL}}
\DeclareMathOperator{\SL}{\mathit{SL}}
\DeclareMathOperator{\PSL}{\mathit{PSL}}

% Theorems
\theoremstyle{plain}
\numberwithin{equation}{section}
\newtheorem{proposition}[equation]{Proposition}
\newtheorem{theorem}[equation]{Theorem}
\newtheorem{lemma}[equation]{Lemma}
\newtheorem{corollary}[equation]{Corollary}
\newtheorem{claim}[equation]{Claim}
\newtheorem{conjecture}[equation]{Conjecture}
\newtheorem{observation}[equation]{Observation}

\theoremstyle{definition}
\newtheorem{definition}[equation]{Definition}
\newtheorem{exercise}[equation]{Exercise}
\newtheorem{question}[equation]{Question}
\newtheorem{problem}[equation]{Problem}

\theoremstyle{remark}
\newtheorem{example}[equation]{Example}
%\newtheorem{hint}[equation]{Hint}
\newtheorem*{remark}{Remark}
%\newtheorem{apology}[equation]{Apology}
%\newtheorem{warning}[equation]{Warning}

% Hyphenation.
\hyphenation{Thurs-ton}

% Noting daily scribes
\newcommand{\scribe}[2]{\marginpar{\small #1\\\texttt{#2}}}

%%% Local Variables: 
%%% mode: latex
%%% TeX-master: "main"
%%% TeX-master: t
%%% End: 
